\chapter*{\begin{center}
    Conclusion générale
\baselineskip=18pt
\end{center}}
\begin{doublespace}
\par

% The main objective of this work is to develop a data preprocessing tool for information retrevial.

% I had to automatize the preprocessing stage in the NLP by developing a tool to preprocess tha data in order to analyse sentiment and recognise the entities in a text databse.

% First, I made a comparative study of existing text preprocessing techniques which allowed me to design two preprocessing tools: One for sentiment analysis and the second for named entity recognition .

% To test our tool, I processed my two databases. By using CRF, SVM and LR I could clearly see the effect of preprocessing on named entity recognition. For the sentiment analysis, the databse and the nature of the models used prevented the tool to increase the accuracy of the models. However, it reduces the training and prediction time. 

% This experience within HENCEFORTH has been very enriching on the technical
% and professional level. It has been an opportunity to expand my knowledge and skills in the field of data science. It was also an opportunity to learn more about teamwork and to become familiar with the world of work.

% In terms of perspectives, the solution we have implemented
%  is still customizable, we can add or remove preprocessing techniques as needed. We can also make our tool more general by adding other options to read different types of databases.

\textbf{Conclusion générale}


Ce projet a été une expérience stimulante et formatrice, marquant une étape significative dans mon parcours professionnel. En tant que projet clé pour REDAL dans le secteur des services publics, notre mission d'intégrer la signature électronique dans le processus de gestion des documents administratifs (DADs) a permis de relever des défis techniques complexes et d'approfondir notre compréhension des dynamiques d'intégration de systèmes.

Tout au long du projet, nous avons mené une analyse approfondie des exigences et des besoins de REDAL, en tenant compte des aspects de sécurité, de conformité, et de performance. En intégrant des technologies avancées comme DocuSign, Node.js, et React.js, et en appliquant des méthodologies agiles, nous avons pu développer une solution robuste et efficace. Ce travail a non seulement amélioré l'efficacité opérationnelle de l'entreprise en automatisant la signature des documents, mais a également fourni une base solide pour une expansion future et une innovation continue dans la gestion des processus administratifs.

L'un des aspects les plus gratifiants de ce projet a été la collaboration étroite avec les différentes parties prenantes chez REDAL. Grâce à une communication régulière et transparente, nous avons pu répondre aux besoins spécifiques de l'entreprise et apporter les ajustements nécessaires tout au long du processus de développement. Cette expérience a également renforcé mes compétences en gestion de projet et en résolution de problèmes, me préparant mieux aux futurs défis dans le domaine de l'intégration de systèmes et de la transformation numérique.

\textbf{Perspectives}

En regardant vers l'avenir, plusieurs initiatives prometteuses pourraient prolonger et enrichir ce travail. Une perspective particulièrement intéressante serait l'implémentation d'un système de monitoring fonctionnel via un tableau de bord pour la direction. Cette avancée permettrait une surveillance en temps réel des performances et des anomalies liées aux processus de signature électronique, assurant ainsi une optimisation continue et une réactivité accrue face aux éventuels dysfonctionnements.

De plus, l'exploration de nouvelles technologies comme l'intelligence artificielle pour l'analyse prédictive et la gestion proactive des systèmes pourrait grandement améliorer l'intégration et la gestion des données au sein de REDAL. Cette démarche ouvrirait la voie à des processus encore plus automatisés et intelligents, capables de répondre aux exigences toujours croissantes en matière de performance et de sécurité.

\textbf{Appréciation Personnelle}

Je tiens à exprimer ma profonde gratitude pour l'opportunité de participer à ce projet d'envergure chez REDAL. Travailler sur un projet aussi crucial pour l'entreprise m'a permis de développer des compétences précieuses et d'acquérir une expérience inestimable. La collaboration avec une équipe de professionnels engagés et la résolution de défis techniques complexes ont été particulièrement enrichissantes. Cette expérience a solidifié mon intérêt pour l'intégration de systèmes et la gestion de projets, et j'ai hâte de continuer à explorer et à contribuer à ce domaine passionnant dans mes futures activités professionnelles.
\end{doublespace}

