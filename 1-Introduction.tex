\baselineskip=18pt

Les objectifs principaux de ce chapitre sont les suivants :
\begin{itemize}
    \item  Introduction à l'Organisation Hôte.
    \item  Aperçu du Projet Client
    \item  Présentation du Projet, comprenant :
    \begin{itemize}
        \item Vue d'ensemble du Contexte Général du Projet
        \item Description de la Problématique
        \item Explication de l'Objectif Principal du Projet
    \end{itemize}
    \item Aperçu de la Gestion de Projet
\end{itemize}
\newpage
\section{Introduction}
\quad Dans ce chapitre, nous présenterons le contexte général du projet, à un niveau organisationnel en présentant l'organisation hôte et ses services, et à un niveau contextuel qui reflète la problématique et les objectifs du projet ainsi que la méthodologie de travail pendant son développement.

\section{Présentation de l’organisme d’accueil}
\quad Cette section offre un aperçu succinct de l'identité et des activités de l'organisme d'accueil. Elle vise à présenter l'entreprise chargée d'accueillir le projet, en mettant l'accent sur les éléments cruciaux et les domaines opérationnels. L'objectif est de garantir aux lecteurs une compréhension claire de l'organisation, de ses principales caractéristiques et de l'étendue de ses opérations.

\medskip
\begin{figure}[!h]
    \centering
    \begin{subfigure}[b]{0.45\textwidth}
        \centering
        \includegraphics[width=\textwidth]{images-REDAL/veolia lgo.png}
        \caption{Groupe VEOLIA's logo}
        \label{fig:2a}
    \end{subfigure}
    \hfill
    \begin{subfigure}[b]{0.36\textwidth}
        \centering
        \includegraphics[width=\textwidth]{images-REDAL/redal logo.png}
        \caption{REDAL's logo}
        \label{fig:2b}
    \end{subfigure}
    \caption{Logos de l'organisme d'accueil}
    \label{fig:2}
\end{figure}


\subsection{Présentation du groupe VEOLIA}

Depuis sa création en 1914 et avec une présence dans 40 pays étalés sur 5 continents dont 15 
en Afrique, le groupe place le développement durable au cœur de ses préoccupations, et ce à 
travers ses trois activités principales : La gestion de l’eau, des déchets et de l’énergie, mais 
aussi grâce à ses trois autres activités complémentaires à savoir le développement de l’accès 
aux ressources, la préservation des ressources disponibles et leur renouvellement. 
Avec un chiffre d’affaire de 21.125 milliards d’euros réalisé en 2017, la multinationale 
française VEOLIA compte en son actif 96 millions d’habitants en eau portable, 62 millions 
en assainissement, la production de 55 millions de mégawatheures et 47 millions de tonnes 
de déchets valorisé. 

\medskip
\begin{figure}[!h]
    \centering
    \includegraphics[width=11cm]{images-REDAL/Screenshot 2024-07-17 214730.png}
    \caption{VEOLIA en numéros}
    \label{fig:l}
\end{figure}
\newpage
Veolia au Maroc, c’est 16 ans d’investissement au profit des populations de Rabat - Salé, 
avec 5327 MDH de cumul des investissements réalisés depuis 2002, dont l’amélioration des 
conditions d’accueil dans les agences commerciales (24 agences commerciales et 4 agences 
mobiles), la mise en œuvre d’un programme de branchement sociaux dans les quartiers non 
structurés, et la mise en place d’innovation au services de leurs clients tels que de nouvelles 
solutions alternatives de paiement, un nouveau réseau de proximité , le lancement de la 
démarche engagement de services, la mise en œuvre des bornes d’information en agences et 
bien d’autres innovations que nous découvrirons à travers la filiale de Veolia Maroc : Redal. 

\medskip
\begin{figure}[!h]
    \centering
    \includegraphics[width=10cm]{images-REDAL/Screenshot 2024-07-17 214524.png}
    \caption{Cumul des 16 ans d’investissements de VEOLIA Maroc}
    \label{fig:2}
\end{figure}


Des investissements colossaux qui ont néanmoins permis au groupe de s'imposer dans le 
marché marocain de distribution de l'eau et l'électricité et de l'assainissement lui permettant 
ainsi d'ouvrir ses deux filiales au Maroc: Amendis et Redal.


\subsection{Présentation de l’activité de REDAL}
Redal est un opérateur privé chargé de la gestion déléguée des services de la distribution 
d’électricité, d’eau potable et d’assainissement liquide. Elle centre ses efforts au profit de 2.2 
millions d’habitants répartis sur 23 communes et arrondissements de Rabat, Salé, Témara, 
Skhirat ainsi que Bouznika et Charrat soit 20\% de la région de Rabat –salé –Kenitra.
\subsubsection{Son Périmètre d’activité}
\medskip
\begin{figure}[!h]
    \centering
    \includegraphics[width=10cm]{images-REDAL/carte_perimetre_0.png}
    \caption{Carte de périmètre d'activité de REDAL}
    \label{fig:2}
\end{figure}
L’entreprise REDAL est présente dans 23 communes et arrondissements de Rabat, Salé, 
Témara, Skhirat ainsi que Bouznika et Charrat. Elle couvre les régions de Rabat –Salé avec 
ses3 métiers de distribution tandis qu’elle s’occupe de la distribution de l’électricité 
uniquement dans la région de Bouznika et la commune de Sidi Taibe de Skhirat. ONEE secharge alors de la distribution de l’eau l’électricité et de l’assainissement dans SEHOUL, Ain 
Ouada, Oum AZZA et El Menzeh.

\subsubsection{Ses Chiffre clés}
Redal veille assidûment à la performance des services dont elle est chargée, à savoir la 
distribution d’électricité, d’eau potable et d’assainissement liquide
Cette dernière compte en son actif :
\begin{itemize}
    \item Près de 1700 collaborateurs 
    \item  2.2 millions d’habitants desservis sur son périmètre d’activité (voir la carte) 
    \item 1185000 clients dont 93\% satisfaits d’après des études de satisfaction menés au 
préalable. 
    \item 24 agences commerciales sur l’ensemble de son périmètre et 178 espaces Jiwar afin de 
servir tous ses clients dans l’ensemble des zones d’activité. 
    \item 4 agences mobiles qui se déplacent en fonction du flux et la demande afin de faciliter le service et réduire la pression sur les agences commerciales. 
    \item 623847 clients en électricité, un réseau d’électricité de 7254 km et taux de déserte 
   93.17\% 
   \item  504423 clients en eau potable, un réseau eau potable de 4124 km et taux de déserte 99.3\% 
   \item Réseau assainissement 2139 km Taux de déserte 93.71\% 
\end{itemize}

Ce qui permet d’Assurer la continuité d’alimentation par l’entretien et la maintenance des 
ouvrages et des réseaux.


\subsection{Mission de l’entreprise}
\subsubsection{REDAL et la relation client :}
A l’image du groupe VEOLIA, l’entreprise Redal met le client au cœur de ses stratégies 
différenciées considérant ainsi la relation client l’une de leurs principales préoccupations et 
veille à mettre les 16 années d’investissements de Redal au Maroc au profit de ses clients que 
l’on peut compter au nombre trois (3) : Les administrations publiques, les institutions 
financières et non financière et les ménages, et ce à travers leurs 10 engagements de service 
(annexe n°1) qui ont pour but de :
\begin{itemize}
    \item Améliorer la prestation des services en insistant sur le produit (engagement n°1) 
l’aménagement des horaires (engagement n°2) et la gestion du temps (engagement 
n°3). 
   \item Favoriser une meilleure relation client en se focalisant sur une réponse rapide aux 
réclamations (sous un délai maximal de 24H), une ligne téléphonique ouverte 24h/24 
et 7j/7, ainsi qu’une communication direct avec et un suivi personnalisé des clients 
(engagements n°4, 5,9 et 10)
   \item Assurer une meilleure gestion de la consommation à travers la présentation de factures 
détaillées, un avertissement en cas de consommation inhabituelle et la proposition de 
conseils pour mieux la maitriser (engagements n°6, 7, 8).

\medskip
\begin{figure}[!h]
    \centering
    \includegraphics[width=13cm]{images-REDAL/Chiffres clès 2022 mètiers.png}
    \caption{Missions de REDAL}
    \label{fig:3}
\end{figure}

\newpage

Afin de pouvoir respecter ses engagements, l’entreprise Redal met à la disposition de ses 
clients, sur place ou à distance, un grand nombre de services pour répondre à toutes sortes de 
demandes et exigences, mais répond aussi aux besoins de sa communauté et ce à travers son 
engagement dans les actions sociétales.
\end{itemize}





%------------------
\newpage
\section{Présentation du projet client}
\subsection{Structure du Projet}
Notre projet s’inscrit dans une initiative de transformation digitale pour l’entreprise REDAL, visant à automatiser et optimiser le processus de signature des documents administratifs DAD par les dirigeants de l’entreprise. Ce projet est conçu pour améliorer l’efficacité, la sécurité et la traçabilité des opérations internes en utilisant l’API REST de DocuSign.

Le projet est subdivisé en deux parties principales, chacune se concentrant sur un aspect spécifique et critique du système global. La première partie se focalise sur la création et l’envoi de mails personnalisés contenant des documents à signer numériquement. Cette partie inclut le développement d'une interface utilisateur intuitive en React.js, permettant aux employés de REDAL de générer des mails et de joindre des documents à signer. Le backend, développé en Node.js avec Express, traite les demandes et utilise l'API DocuSign pour envoyer les documents, en assurant la sécurité et l’intégrité des informations transmises.

La seconde partie du projet concerne la consultation des mails envoyés et le suivi de leur statut via l'API DocuSign. Cela permet aux utilisateurs de REDAL de vérifier l'état des signatures des documents, de savoir s'ils ont été signés ou sont en attente. Le frontend affiche ces informations de manière claire et accessible, tandis que le backend interroge régulièrement l’API DocuSign pour mettre à jour les statuts des documents dans la base de données PostgreSQL.

Chaque partie du projet est essentielle pour la réussite globale, apportant des solutions spécialisées et des améliorations continues qui contribuent à la réalisation des objectifs de REDAL en matière de digitalisation et d'automatisation des processus administratifs. L’utilisation de Docker pour la conteneurisation des applications garantit une déploiement et une gestion simplifiés des environnements, assurant ainsi une plus grande fiabilité et flexibilité des opérations.
\subsection{Build et Run}
Pour ce projet, nous nous chargeons de l’administration de l’infrastructure et des services nécessaires pour la mise en œuvre et la gestion de la solution de signature électronique via l’API DocuSign pour les documents DAD au sein de REDAL, en suivant les volets Build et Run :

\textbf{Build :}

Quand REDAL exprime un nouveau besoin technique ou fonctionnel découlant de l’évolution des besoins métier, notre équipe intervient pour développer de nouvelles fonctionnalités ou adapter les composants existants de l’infrastructure. Cela inclut l’ajout de nouvelles fonctionnalités à l’interface utilisateur pour la génération et l’envoi de mails personnalisés avec des documents à signer, ainsi que l’intégration de nouvelles API ou services si nécessaire.

\textbf{Run :}

Cette étape suit la phase de construction des composants de l’infrastructure. Elle inclut le monitoring continu des services et des applications pour garantir le bon déroulement des opérations et des interactions entre les différentes composantes de notre solution. Des problèmes et incidents peuvent survenir en raison des modifications apportées aux données, aux traitements ou aux accès. Le monitoring vise donc à améliorer continuellement le service. Lorsque REDAL rencontre des problèmes affectant le bon déroulement de ses activités, ces incidents sont signalés via une plateforme de gestion des incidents. Les incidents sont alors assignés à l’équipe appropriée pour une résolution rapide. Les incidents sont classés par niveau de priorité en fonction de leur impact et de leur urgence, ce qui permet de traiter en priorité ceux qui ont un impact significatif sur les activités de l’entreprise.


\newpage
\section{Présentation du Projet}
\quad Dans cette section, nous allons d’abord situer le contexte général du projet. Nous discuterons ensuite de la problématique à laquelle le projet vise à répondre. Enfin, nous définirons l’objectif principal du projet, mettant en lumière son but et sa valeur ajoutée.

\subsection{Contexte Général du Projet :}

REDAL, une entreprise de services publics, s'efforce continuellement d'améliorer l'efficacité et la sécurité de ses processus internes. Dans le cadre de cette initiative, l'entreprise a identifié un besoin crucial de digitalisation et d'automatisation des signatures des documents administratifs (DAD). Traditionnellement, ces documents devaient être signés manuellement par le Directeur Général et le Directeur des Achats, un processus qui non seulement ralentissait les opérations, mais augmentait également le risque d'erreurs et de retards.

Pour répondre à ce besoin, le projet de création d'un site web permettant de générer des mails personnalisés contenant des documents à signer numériquement via l'API REST de DocuSign a été lancé. Ce projet vise à simplifier et sécuriser le processus de signature des documents DAD, en offrant une solution numérique robuste et facile à utiliser pour les employés de REDAL.


\subsection{Problématique}
Le processus actuel de signature des documents DAD chez REDAL est manuel et chronophage. Les documents doivent être physiquement transférés entre les bureaux, ce qui entraîne des délais considérables et des risques accrus de perte ou de falsification des documents. De plus, l'approbation et la signature de ces documents par le Directeur Général et le Directeur des Achats sont souvent retardées, affectant l'efficacité globale de l'entreprise.

Cette situation crée une série de problématiques :

\begin{itemize}

    \item \textbf{Temps et ressources :}
Le processus manuel est long et consomme beaucoup de ressources humaines.

\item \textbf{Sécurité et traçabilité :}
Les documents papier sont susceptibles de se perdre ou d'être altérés, et il est difficile de suivre l'état d'avancement de chaque document.

\item \textbf{Réactivité :} 
Les retards dans la signature des documents peuvent entraîner des ralentissements dans les opérations de l'entreprise.

\end{itemize}
\subsection{Objectif principal du projet}
L'objectif principal de ce projet est de digitaliser et automatiser le processus de signature des documents DAD chez REDAL en utilisant l'API REST de DocuSign. La solution proposée consiste à développer un site web qui permet aux employés de REDAL de générer des mails contenant des documents à signer et de les envoyer directement via le site. Les utilisateurs peuvent consulter les mails envoyés et suivre en temps réel le statut des documents (signé, en attente, etc.) via l'interface utilisateur. En utilisant l'API DocuSign, le projet garantit la sécurité des signatures électroniques et la conformité aux normes légales. La valeur ajoutée de ce projet réside dans l'amélioration de l'efficacité opérationnelle de REDAL, la réduction des risques de sécurité et l'amélioration de la traçabilité des documents. En facilitant et en sécurisant le processus de signature des documents, ce projet contribue de manière significative à la modernisation des processus internes de REDAL, rendant l'entreprise plus agile et réactive face aux besoins métiers.


\section{Conduite du projet}
Après avoir présenté le contexte, la problématique et les objectifs du projet, nous allons, dans cette section, analyser les différentes phases, méthodologies et pratiques de gestion de projet appliquées pour en assurer la réussite du projet.

\subsection{Méthodologie de travail}

Le terme SCRUM fait référence à la mêlée de rugby. C'est une méthode astucieuse de management qui facilite la gestion des aspects humains d'un projet, en particulier la répartition des ressources humaines. Les projets utilisant la méthode agile sont divisés en plusieurs phases de travail, appelées "sprints", qui durent généralement entre une et quatre semaines. Ces sprints permettent aux membres de l'équipe de mieux planifier les étapes suivantes du développement du projet, tout en assurant un suivi et une évaluation régulière des progrès. Ils offrent également la possibilité d'adapter ou de réorienter le projet si nécessaire.\cite{2}

\begin{figure}[H]
\begin{center}
\includegraphics[width=14cm]{image/ch1/scrum.png}
\end{center}
\caption{Méthode Scrum}
\end{figure}


Dans ce projet, nous avons adopté une approche hybride pour le suivi et la gestion du travail, adaptée à nos contraintes spécifiques. Chaque semaine, nous avons une réunion présentielle à l'entreprise pour discuter des avancements et des plans futurs. En complément, un suivi quotidien des progrès a été mis en place, bien que sans réunion formelle quotidienne. Cette méthode nous permet de maintenir une communication continue et efficace tout en offrant la flexibilité nécessaire pour s'adapter aux besoins individuels et aux imprévus du projet. L'interaction hebdomadaire en personne renforce la collaboration et la cohésion d'équipe, tandis que le suivi quotidien assure que les objectifs sont atteints de manière progressive et que les obstacles sont rapidement identifiés et résolus.

\subsubsection{Les avantages de la méthode SCRUM}
\begin{itemize}
    \item \textbf{Personnel engagé} : SCRUM encourage l'engagement du personnel en les impliquant activement dans la définition des activités et des horaires, ce qui accroît leur motivation et implication.
    \item \textbf{Vue d'ensemble améliorée} : SCRUM permet à tous les membres de l’équipe de comprendre de manière homogène les objectifs et les tâches à accomplir, offrant ainsi une meilleure vue d’ensemble du projet.
    \item \textbf{Mise à jour des priorités} : Avec SCRUM, le client dispose d'une flexibilité dans la définition et l'évolution des priorités et des séquences d’activités, permettant une adaptation aux besoins changeants.
    \item \textbf{Accent sur la qualité du produit} : SCRUM met l’accent sur la fourniture d’un service de valeur au client plutôt que sur le respect strict des délais, favorisant ainsi la qualité du produit.
    \item \textbf{Communication renforcée} : SCRUM encourage une communication régulière et transparente au sein de l’équipe, facilitant une collaboration efficace et une résolution rapide des problèmes.
    \item \textbf{Réduction des risques} : Grâce à des itérations courtes et des revues régulières, SCRUM permet de détecter et de résoudre les problèmes plus rapidement, réduisant ainsi les risques liés au projet.
\end{itemize}


\subsubsection{Répartition des rôles dans SCRUM}
La méthodologie Scrum repose sur trois rôles distincts pour notre projet :
\begin{itemize}
    \item \textbf{Le Scrum Master} : Il est responsable de la compréhension, de l’adhésion et de la mise en œuvre de la méthode SCRUM. Le Scrum Master facilite la communication au sein de l’équipe et cherche à maximiser sa productivité. Il veille également au respect des principes et des valeurs de SCRUM.
    \item \textbf{Le Product Owner} : Il porte la vision du produit à réaliser et travaille en interaction avec l’équipe de développement. Le Product Owner établit les priorités des fonctionnalités à développer ou à corriger et valide les fonctionnalités terminées. Il
    est responsable de la gestion du product backlog et s’assure que les besoins du client sont correctement pris en compte.
    \item \textbf{L’équipe de développement} : Elle est chargée de transformer les besoins définis par le Product Owner en fonctionnalités utilisables. Les décisions au sein de l’équipe de développement sont prises collectivement, sans notion de hiérarchie.
    
\end{itemize}

\subsubsection{Matrice de participation pour l’équipe Integration Platform}

\begin{table}[H]
\centering
\begin{tabular}{|l|p{10cm}|}
\hline
\textbf{Rôle} & \textbf{Personnes affectées} \\ \hline
Product Owner &  REDAL's Adminstration\\ \hline
Scrum Master & Mr. ELHAMDI Ahmed\\ \hline
%Chefs de projet & M. TAZI Driss - Mme Bakkar Rajaa \\ \hline
Équipe de projet & Youssef ELGARCH\\ \hline
\end{tabular}
\caption{Répartition des rôles dans l'équipe}
\label{tab:roles}
\end{table}
La table 1.1 présente l’équipe SCRUM de Stream Integration Platform pour le projet. Nous retrouvons un product owner ...

\subsubsection{Les différents événements du SCRUM}
Dans une méthodologie agile, divers événements clés sont organisés de manière régulière pour promouvoir la collaboration, la transparence et l'adaptabilité. 

\subsection{Planification des réunions}


\subsection{Planification du projet}
Avant de débuter le projet, il est crucial de procéder à une planification minutieuse de sa mise en œuvre. L'objectif de cette planification est double : d'une part, découper le projet en plusieurs phases intermédiaires afin de mieux estimer sa durée totale et les ressources nécessaires, et d'autre part, valider séquentiellement sa conformité avec les besoins exprimés.

Suite aux réunions avec l'encadrant, il a été recommandé de diviser le projet en sprints selon la méthode SCRUM. Ces sprints impliquent des activités clés telles que la réunion de lancement, la planification du projet, et l'échantillonnage des exigences du produit. En accord avec ces principes, le projet a été mis en œuvre en suivant un planning basé sur un diagramme de Gantt, tel qu'illustré dans la figure 1.11. Ce diagramme présente de manière chronologique les tâches réalisées, classées dans un ordre précis.

\newpage
  \begin{figure}[H]
    \begin{center}
    \rotatebox{90}{\includegraphics[width=22cm , height = 6cm]{images-REDAL/hhhhhhhhhhhhhhhhhh.png}}
    \end{center}
    \caption{Diagramme de Gantt}
\end{figure}

La mise en place de ce planning a été cruciale pour la gestion du temps et l'exécution efficace du projet. Grâce à cette planification, nous avons pu établir des échéances claires pour chaque phase du projet et garantir que les livrables étaient produits dans les délais impartis. 



Le planning a été élaboré de manière à intégrer les différentes étapes et activités nécessaires à la réalisation du projet. Chaque phase a été minutieusement planifiée, définissant les objectifs à atteindre, les tâches à accomplir, et les délais à respecter.
    

\section{Conclusion}

\quad Tout au long de ce chapitre, nous avons exploré le cadre global du projet. Nous avons débuté par une introduction à l'entreprise d'accueil et au projet client pour lequel notre projet a été réalisé. Ensuite, nous avons examiné le contexte du projet, suivi de la méthodologie de gestion de projet adoptée et de la planification mise en place. Dans le chapitre à venir, nous nous pencherons sur l'état de l'art.








