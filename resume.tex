\chapter*{\begin{center}
    Résumé
\baselineskip=18pt
\end{center}}

\quad Le présent document constitue la synthèse du travail réalisé dans le cadre de mon stage de fin d’année effectué au sein de REDAL dans les domaines des Smart Cities. Ma mission était de développer une interface entre les systèmes internes de REDAL et la solution DocuSign pour intégrer des signatures électroniques dans le processus des DAD (Demandes d'Achat Direct).

Ce projet a été géré en utilisant l’approche agile, ce qui nous a permis d’être très réactifs aux nombreux paramètres changeant dans nos tâches. Grâce à cette méthodologie, nous avons pu effectuer des itérations rapides et recevoir des retours fréquents, améliorant ainsi la qualité et la pertinence de notre solution.

Le développement a impliqué l'utilisation de technologies modernes telles que Node.js pour le backend, React.js pour le frontend, et l'API de DocuSign pour la gestion des signatures électroniques. La mise en place d'une base de données PostgreSQL a également été nécessaire pour stocker et gérer les données des documents et des signataires.

En plus de l'aspect technique, ce projet m'a permis de développer des compétences en gestion de projet et en travail d'équipe, tout en approfondissant mes connaissances en matière de sécurité et de gestion des données sensibles.

En conclusion, ce stage a été une expérience enrichissante et formatrice, qui m'a permis de contribuer à un projet innovant et de développer des compétences techniques et professionnelles précieuses.
\vspace{0.5cm}


\noindent\rule[2pt]{\textwidth}{0.5pt}

{\textbf{Mots clés :}}
DocuSign,
Villes intelligentes,
Signatures électroniques,
Demandes d'achat directes (DAD),
API DocuSign,
Développement web,
Automatisation des processus
\\
\noindent\rule[2pt]{\textwidth}{0.5pt}

