\documentclass[12pt,french,latin.medieval,titlepage]{report}
\usepackage{Preamble}
\usepackage[utf8]{inputenc}
\usepackage[T1]{fontenc}
\usepackage[french]{babel}
\usepackage[light, largesmallcaps]{kpfonts}
\usepackage{lipsum}
\usepackage{array}


\newcolumntype{R}[1]{>{\centering\arraybackslash }b{#1}}
\newcolumntype{L}[1]{>{\centering\arraybackslash }b{#1}}
\newcolumntype{C}[1]{>{\centering\arraybackslash }b{#1}}



\begin{document}
\baselineskip=18pt

\subfile{firstpage}


\subfile{Acknowledgments}

\subfile{resume}

\subfile{abstract}





\restoregeometry

\thispagestyle{numberonly}
\subfile{abreviations}
\newpage
\baselineskip=20pt
\addcontentsline{toc}{chapter}{Liste des tableaux}
\listoftables
\newpage
\addcontentsline{toc}{chapter}{Table des figures}
\listoffigures

\newpage
\baselineskip=20pt
\addcontentsline{toc}{chapter}{Table des matières}

\tableofcontents
\subfile{introduction}
\chapter{Contexte général du projet}
\subfile{1-Introduction}

\chapter{Etude et Analyse}
\subfile{2-Methods}

\chapter{Étude conceptuelle et architecturale}
\subfile{3-Results}

\chapter{Réalisation et Mise en œuvre}
\subfile{4-Discussion}






%\chapter{Project realization and results}
%\subfile{5-Conclusion}

\addcontentsline{toc}{chapter}{Conclusion Générale}
\subfile{generalconclusion}

%\addcontentsline{toc}{chapter}{Glossary}
%\subfile{Glossary}

\nocite{*}
%choix du style de la biblio

\newpage
\bibliographystyle{plain}
\bibliography{bibliography.bib}
% voir wiki pour plus d'information sur la syntaxe des entrées d'une bibliographie
\end{document}

