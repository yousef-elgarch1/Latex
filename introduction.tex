\baselineskip=18pt
\addcontentsline{toc}{chapter}{Introduction Générale}

\chapter*{\begin{center}
    Introduction générale
\end{center}}

 L’intégration des technologies numériques dans les processus administratifs représente un enjeu majeur pour les entreprises modernes, cherchant à optimiser leur efficacité et à sécuriser leurs opérations. REDAL, acteur clé dans le secteur de la gestion des services urbains, se tourne vers des solutions innovantes pour améliorer ses processus internes, notamment les procédures d’approvisionnement.

Mon stage chez REDAL m’a permis de participer à un projet stratégique visant à intégrer la signature numérique dans les documents DAD via DocuSign. Ce projet est essentiel pour simplifier et sécuriser le processus de signature des documents d’achat, en offrant deux principales fonctionnalités : permettre la signature directe des documents via l’application web et envoyer des courriels contenant les documents à signer via DocuSign.

Ce rapport synthétise le déroulement de mon travail sur ce projet et est structuré en quatre chapitres :

\begin{itemize}
    

\item \textbf{Chapitre 1 : Présentation de REDAL et du Projet} Ce chapitre comprend une description de l’organisme d’accueil "REDAL", du projet d’intégration de la signature numérique, ainsi qu’une présentation du cadre général de ce projet. Il expose la problématique, les objectifs, et aborde la méthodologie appliquée pour assurer le bon déroulement de mon travail avec une planification de projet.

\item \textbf{Chapitre 2 : Concepts Clés et Analyse des Besoins} Ce chapitre explore en détail les concepts clés liés à la signature numérique et à la gestion des documents administratifs, ainsi qu’une analyse approfondie des besoins spécifiques de REDAL pour ce projet.

\item \textbf{Chapitre 3 : Étude Conceptuelle et Architecturale} Ce chapitre décrit l’étude conceptuelle et architecturale du projet, incluant le choix des technologies, la conception de l’application web, et le processus de déploiement.

\item \textbf{Chapitre 4 : Réalisation du Projet} Ce chapitre relate la phase de réalisation du projet, détaillant les étapes de développement, les défis rencontrés, et les solutions apportées pour intégrer efficacement DocuSign dans les documents DAD.

\quad
Enfin, je conclurai par un résumé des principaux enseignements tirés de mon stage et de leur pertinence pour mes aspirations professionnelles futures, ainsi que des propositions pour l'amélioration continue du processus d’intégration de la signature numérique chez REDAL

\end{itemize}

